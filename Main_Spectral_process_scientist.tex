%%%%%%%%%%%%%%%%%
% This is an sample CV template created using altacv.cls
% (v1.3, 10 May 2020) written by LianTze Lim (liantze@gmail.com). Now compiles with pdfLaTeX, XeLaTeX and LuaLaTeX.
% This fork/modified version has been made by Nicolás Omar González Passerino (nicolas.passerino@gmail.com, 15 Oct 2020)
%
%% It may be distributed and/or modified under the
%% conditions of the LaTeX Project Public License, either version 1.3
%% of this license or (at your option) any later version.
%% The latest version of this license is in
%%    http://www.latex-project.org/lppl.txt
%% and version 1.3 or later is part of all distributions of LaTeX
%% version 2003/12/01 or later.
%%%%%%%%%%%%%%%%

%% If you need to pass whatever options to xcolor
\PassOptionsToPackage{dvipsnames}{xcolor}

%% If you are using \orcid or academicons
%% icons, make sure you have the academicons
%% option here, and compile with XeLaTeX
%% or LuaLaTeX.
% \documentclass[10pt,a4paper,academicons]{altacv}

%% Use the "normalphoto" option if you want a normal photo instead of cropped to a circle
% \documentclass[10pt,a4paper,normalphoto]{altacv}

\documentclass[10pt,a4paper,ragged2e,withhyper]{altacv}

%% AltaCV uses the fontawesome5 and academicons fonts
%% and packages.
%% See http://texdoc.net/pkg/fontawesome5 and http://texdoc.net/pkg/academicons for full list of symbols. You MUST compile with XeLaTeX or LuaLaTeX if you want to use academicons.

% Change the page layout if you need to
\geometry{left=1.2cm,right=1.2cm,top=1cm,bottom=1cm,columnsep=0.75cm}

% The paracol package lets you typeset columns of text in parallel
\usepackage{paracol}
\usepackage{framed}
% Change the font if you want to, depending on whether
% you're using pdflatex or xelatex/lualatex
\ifxetexorluatex
  % If using xelatex or lualatex:
  \setmainfont{Roboto Slab}
  \setsansfont{Lato}
  \renewcommand{\familydefault}{\sfdefault}
\else
  % If using pdflatex:
  \usepackage[rm]{roboto}
  \usepackage[defaultsans]{lato}
  % \usepackage{sourcesanspro}
  \renewcommand{\familydefault}{\sfdefault}
\fi

%----- LIGHT MODE -----
\definecolor{SlateGrey}{HTML}{2E2E2E}
\definecolor{LightGrey}{HTML}{666666}
\definecolor{PrimaryColor}{HTML}{001F5A}
\definecolor{SecondaryColor}{HTML}{0039AC}
\definecolor{ThirdColor}{HTML}{F3890B}
\definecolor{BackgroundColor}{HTML}{FFFFFF}
\definecolor{shadecolor}{HTML}{F2F2F2}
\colorlet{name}{PrimaryColor}
\colorlet{tagline}{PrimaryColor}
\colorlet{heading}{PrimaryColor}
\colorlet{headingrule}{ThirdColor}
\colorlet{subheading}{SecondaryColor}
\colorlet{accent}{SecondaryColor}
\colorlet{emphasis}{SlateGrey}
\colorlet{body}{LightGrey}
\pagecolor{BackgroundColor}   

% ----- DARK MODE -----
% \definecolor{BackgroundColor}{HTML}{242424}
% \definecolor{SlateGrey}{HTML}{6F6F6F}
% \definecolor{LightGrey}{HTML}{ABABAB}
% \definecolor{PrimaryColor}{HTML}{3F7FFF}
% \colorlet{name}{PrimaryColor}
% \colorlet{tagline}{PrimaryColor}
% \colorlet{heading}{PrimaryColor}
% \colorlet{headingrule}{PrimaryColor}
% \colorlet{subheading}{PrimaryColor}
% \colorlet{accent}{PrimaryColor}
% \colorlet{emphasis}{LightGrey}
% \colorlet{body}{LightGrey}
% \pagecolor{BackgroundColor}

% Change some fonts, if necessary
\renewcommand{\namefont}{\Huge\rmfamily\bfseries}
\renewcommand{\personalinfofont}{\small\bfseries}
\renewcommand{\cvsectionfont}{\LARGE\rmfamily\bfseries}
\renewcommand{\cvsubsectionfont}{\large\bfseries}

% Change the bullets for itemize and rating marker
% for \cvskill if you want to
\renewcommand{\itemmarker}{{\small\textbullet}}
\renewcommand{\ratingmarker}{\faCircle}

%% sample.bib contains your publications
%% \addbibresource{sample.bib}

\begin{document}
    
    \name{Thibault Tabarin} 
    \tagline{Data and Image Processing Scientist}
    %% You can add multiple photos on the left or right
    %%\photoL{4cm}{john-doe}
    
    \personalinfo{
        \email{thibault.tabarin@gmail.com}\\
        \phone{720-643-9073}\\
        \location{Boulder, CO}\\
        
        \href{https://www.linkedin.com/in/thibault-tabarin-995b0322}{\faLinkedin Linkedin} 
        \href{https://github.com/thibaulttabarin}{\faGithub Github}
        
        %\dev{johnDoe}
        %\homepage{nicolasomar.me}
        %\medium{nicolasomar}
        %% You MUST add the academicons option to \documentclass, then compile with LuaLaTeX or XeLaTeX, if you want to use \orcid or other academicons commands.
        % \orcid{0000-0000-0000-0000}
        %% You can add your own arbtrary detail with
        %% \printinfo{symbol}{detail}[optional hyperlink prefix]
        % \printinfo{\faPaw}{Hey ho!}[https://example.com/]
        %% Or you can declare your own field with
        %% \NewInfoFiled{fieldname}{symbol}[optional hyperlink prefix] and use it:
        % \NewInfoField{gitlab}{\faGitlab}[https://gitlab.com/]
        % \gitlab{your_id}
    }
    
    \makecvheader

    %% Depending on your tastes, you may want to make fonts of itemize environments slightly smaller
    % \AtBeginEnvironment{itemize}{\small}
    
    %% Set the left/right column width ratio to 6:4.
    %\columnratio{0.25}

    % Start a 2-column paracol. Both the left and right columns will automatically
    % break across pages if things get too long.
    \columnratio{0.25}
    


    \begin{paracol}{2}
        
        % ----- EXPERTISE -----
        \cvsection{Skills}
        %\begin{shadedcvbox}
            \cvsubsection{Expertise}
                \cvtag{Machine learning}
                
                \cvtag{Deep learning}
                
                \cvtag{Computer vision}
                \cvtag{CNN}
                %\cvtag{Encoder-Decoder}
                
                %\cvtag{Segmentation}
                
                \cvtag{Clustering}
                %\cvtag{Markov chain}
                %\cvtag{Bayesian modeling}
                
                %\cvtag{Expectation Maximization}
                
                %\cvtag{Applied mathematics}
                \medskip
                
                \cvtag{Microscopy}
                %\cvtag{LSM}
                
                %\cvtag{FCS/Lifetime Microscopy}
                
                %\cvtag{Super-resolution microscopy}
                \cvtag{Mass Spectrometry}
                
                \cvtag{Spectroscopy}
                %\cvtag{Optics}
                
                \cvtag{Instrumental Physics}
                
                \cvtag{Biophysics}
                \cvtag{Biology}
                
                
             \cvsubsection{Laboratory Practice}           \cvtag{Electrospray}
                \cvtag{Ion Trap}
                
                \cvtag{Photodissociation}
                
                \cvtag{Optics}
                \cvtag{Laser}
                
                \cvtag{Sample preparation}
                
                \cvtag{Immunostaining} 
    
                \cvtag{PDMS microcavity}
                
                \cvtag{Microscope maintenance}
                
        % ----- EXPERTISE -----
        
        % ----- PROGRAMMING -----
            \cvsubsection{Programming Languages}
                \cvtag{Python}
                \cvtag{Matlab}
                %\cvtag{SQL}
                %\cvtag{C++}
                %\cvtag{Bash}
                %\cvtag{Skimage}
                \cvtag{OPENCV}
                \cvtag{Pytorch}
                \cvtag{Git}
                \cvtag{Linux}
                


                
            \cvsubsection{Communication}
                \cvtag{Public speaking}\\
                \cvtag{Technical writing}\\
                \cvtag{Scientific writing}\\
                \cvtag{Code documentation}\\
                
            
            %% Yeah I didn't spend too much time making all the
            %% spacing consistent... sorry. Use \smallskip, \medskip,
            %% \bigskip, \vpsace etc to make ajustments.
            \smallskip
        % ----- PROGRAMMING -----
        
        % ----- PUBLICATION -----
        \cvsubsection{Publication}
        30 peer-reviewed articles.  
        See Accomplishment on  \href{https://www.linkedin.com/in/thibault-tabarin-995b0322}{\faLinkedin}. 
        % ----- PUBLICATION -----
        
        %
        % ----- REFERENCES -----
        \cvsubsection{References}
            \cvreftb{Jeremie Rossy}{https://www.linkedin.com/in/jeremie-rossy/}
            \smallskip
            \cvreftb{Siawoosh Mohammadi}{https://www.linkedin.com/in/siawoosh-mohammadi-646252150/}
            \smallskip
            \cvreftb{Sophie Pageon}{https://www.linkedin.com/in/sophie-pageon-048bb156/}
            
            %\end{shadedcvbox} 
            
            \smallskip
           %\end{shaded}
           
        % ----- REFERENCES -----
        
        % ----- MOST PROUD -----
        % \cvsection{Most Proud of}
        
        % \cvachievement{\faTrophy}{Fantastic Achievement}{and some details about it}\\
        % \divider
        % \cvachievement{\faHeartbeat}{Another achievement}{more details about it of course}\\
        % \divider
        % \cvachievement{\faHeartbeat}{Another achievement}{more details about it of course}
        % ----- MOST PROUD -----
        
        % \cvsection{A Day of My Life}
        
        % Adapted from @Jake's answer from http://tex.stackexchange.com/a/82729/226
        % \wheelchart{outer radius}{inner radius}{
        % comma-separated list of value/text width/color/detail}
        % \wheelchart{1.5cm}{0.5cm}{%
        %   6/8em/accent!30/{Sleep,\\beautiful sleep},
        %   3/8em/accent!40/Hopeful novelist by night,
        %   8/8em/accent!60/Daytime job,
        %   2/10em/accent/Sports and relaxation,
        %   5/6em/accent!20/Spending time with family
        % }
        
        % use ONLY \newpage if you want to force a page break for
        % ONLY the current column
        \newpage
        
        %% Switch to the right column. This will now automatically move to the second
        %% page if the content is too long.
        
        \switchcolumn

        
        % ----- EXPERIENCE -----
        \cvsection{Experience}
            \cvevent{Researcher: Microscopy and deep learning }{| University hospital, UKE}{March 2018 -- August 2019}{Hamburg, Germany}
            \begin{itemize}
                \item Developed algorithms for the \highlight{automatic detection} of millions of axons in white matter (brain) from Electron and Light Microscopy images. The final outcome was the axons morphology (such as sizes) in specific parts of the brain, which are linked to diseases such as Alzheimer.
                \item Built, trained and used an \highlight{encoder-decoder based CNN} (U-net) for semantic and instance \highlight{segmentation}.Achieved an  accuracy of 87\%.
                %\item Implemented a Bayesian approximation for \highlight{uncertainty quantification} and \highlight{active learning}. Each image with a low uncertainty score was manually reanalysed and used to retrain the CNN and improve model generalization. 
                \item Collaborated alongside a team of
                Neuroscientists and biologists to result in a conference paper publication.
            \end{itemize}
            \divider
            
            \cvevent{Researcher: Microscopy and machine learning }{| Univeristy UNSW}{April 2012 -- June 2016}{Sydney, Australia}
            \begin{itemize}
            
            	   	%\item Used machine learning algorithms and super-resolution microscopy (Zeiss) to \highlight{detect} and \highlight{quantify protein clusters} in the membrane of immune cells (T lymphocytes).

	   		        %\item Implemented  \highlight{clustering}, \highlight{classification} and \highlight{morphology characterization} software.
	   		        
	   		        \item Used machine learning algorithms and super-resolution microscopy (Zeiss) to detect and quantify protein clusters in the membrane of immune cells (T lymphocytes).
	   		        \item Implemented  clustering, classification and morphology characterization software.
	   		        \item Prepared samples and performed the acquisitions and analysed the data.
	   		        
	   		        \item Supervised 2 PhD students. Published 4 peer-reviewed articles.
                
            \end{itemize}
            \divider
            
            \cvevent{Postdoctoral Fellow: Microscopy and diffusion }{| Univeristy DCU}{Sept 2009 -- March 2012}{Dublin, Ireland}
            \begin{itemize}
                \item Quantified diffusion properties of integrin proteins and lipids in artificial cell membranes using fluorescence correlation spectroscopy nand lifetime fluorescence microscopy.
                \item Used Picosecond laser, time-gated microscopy to acquire single molecule signal and used function correlation to measure the diffusion parameters and density of proteins and lipids.
                \item Prepared the samples using supported lipid bilayers. Inserted the proteins in bilayers.
                
             \end{itemize}            
             \divider 
             
            \cvevent{Postdotoral Fellow: Microscopy and single molecule }{| Univeristy SNU}{Sept 2008 -- august 2009}{Seoul, South Korea}
            \begin{itemize}
                \item Built a three colors fluorescence resonance energy transfer microscope to measure the diffusion and dynamic of single molecule DNA strands in liquid phase.     
            \end{itemize}
        % ----- EXPERIENCE -----
        

        
        % % ----- PROJECTS -----
        % \cvsection{Projects}
        %     \cvevent{Project 1 }{\cvrepo{| \faGithub}{https://github.com/user/repo}\cvrepo{| \faGlobe}{https://repo-demo.com/}}{Mm YYYY -- Mm YYYY}{}
        %     \begin{itemize}
        %         \item Item 1
        %         \item Item 2
        %     \end{itemize}
        %     \divider
            
        %     \cvevent{Project 1 }{\cvrepo{| \faGitlab}{https://gitlab.com/user/repo}\cvrepo{| \faGlobe}{https://repo-demo.com/}}{Mm YYYY -- Mm YYYY}{}
        %     \begin{itemize}
        %         \item Item 1
        %         \item Item 2
        %     \end{itemize}
        % ----- PROJECTS -----
        % ----- EDUCATION -----
        \cvsection{Education}
            \cvevent{PhD in Physical Chemistry }{| Univeristy of Lyon}{Sept 2004 -- Sept 2008}{Lyon, France}
            \begin{itemize}
                \item Dynamics and optical properties of biomolecules in gas phase : \highlight{Mass Spectrometry} and \highlight{Spectroscopy}
            \end{itemize}
            \medskip
            
            \cvevent{Master }{| University of Lyon}{Sept 2002 -- Sept 2004}{Lyon, France}
            \begin{itemize}
                \item Physics : Photons, Optics, Molecules and Atoms
            \end{itemize}
        % ----- EDUCATION -----        
        
        
    \end{paracol}
%    \newpage
%    \cvsection{Bibliographie}
    
    
\end{document}
